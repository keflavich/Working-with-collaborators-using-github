Authorea is an \textit{online} collaborative editor, but \textbf{working offline is possible}, as long as you get acquainted with the versioning control technologies we use to save and track your documents: \href{http://git-scm.com/}{Git} and \href{http://github.com/}{GitHub}. This how-to guide will assume you have some basic knowledge of Git and that you have a Github account already. 

\textbf{Step 1. Create an Authorea document}. If you have not done so already, go ahead and create a new article. You can start with a template, a blank article, or even an existing article. There are two things to keep in mind at this stage:

\begin{enumerate}
\item \textit{Public or Private?} If you want to work privately (the document will be only available to you and your coauthors), then you have to make sure both your Authorea document AND your Github repository) are \textbf{private}.
\item \textit{Document Title}. The project title of your Authorea document AND your project title of your Github repository have to be \textbf{identical}. So, when creating a new document, make sure you pick a title with only alphanumeric values and dashes; Git and Github do not like spaces or special characters. Example of a good name? \verb|authorea-paper-1|. If you have an existing article, make sure to rename it so that it complies with this style.
\end{enumerate}

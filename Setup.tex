Authorea is an \textit{online} collaborative editor, but \textbf{working offline is possible}, as long as you get acquainted with the versioning control system we use to save and track your documents: \href{http://git-scm.com/}{Git} and \href{http://github.com/}{GitHub}. This how-to guide will assume you have some basic knowledge of Git and that you have a Github account already. 

\textbf{Step 1. Create an Authorea document}. If you have not done so already, go ahead and create an article. You can start with a template, a blank article, or even an existing article. There are two things to keep in mind at this stage:

\begin{enumerate}
\item \textit{Public or Private?} If you want to work privately (the document will be only available to you and your coauthors), then you have to make sure both your Authorea document AND your Github repository) are PRIVATE.
\item \textit{Document Title}. The project title of your Authorea document AND your project title of your Github repository have to be IDENTICAL. So, when creating a new document, make sure you pick a title with only alphanumeric values and dashes; Git and Github do not like spaces or special characters. Examples of good names are \verb|authorea-test|, \verb|authorea-paper-1|, etc. If you have an existing article, make sure to rename it so that it complies with this.
\end{enumerate}

\textbf{Step 2. Create a Github repository}. Log in to your Github account and create a new repository. Remember to name it exactly as the Authorea document (e.g., \verb|authorea-paper-1|) and set it to the same availability (Public vs Private) as the Authorea document.

\textbf{Step 3. Link the Authorea and Github repositories}. Go back to your Authorea document and click on the \verb|Settings| icon in the top toolbar. Then click \verb|Advanced Settings|, and scroll to the bottom and click on the Github integration button \verb|Setup a deploy key|. \textit{Note: you need to have admin provileges to see the Github integration button. Ask your first author to make you admin, if you're not.} 

A screen appears with a number of technical options. What matters to you is at the very top. Click on \verb|Do this automatically|. 




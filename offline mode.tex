\textbf{Step 5. Working offline}. If you navigate back to your Github repository, you will see that all the basic Authorea files (bibliography folder, figures folder, etc) are now in the repo. You can now work offline using your editor of choice and push/pull to/from this Github repo to update the Authorea article. 

\textbf{NOTE:} Most users who write in LaTeX will want to compile their articles from time to time when working offline. We advise users not to change the basic structure of the repository. For example, the file \verb|layout.md| is crucial as it instructs Authorea what files to render. If you delete it, the Authorea article will break! In order to build your article locally when working offline we suggest to use \href{https://gist.github.com/eteq/1d7b138b9c4e80231f6f}{this Python script}\footnote{You can download it directly from this url: \\ \url{https://gist.github.com/eteq/1d7b138b9c4e80231f6f/raw/d535a5c0d7970fd3fd277e0902cec9b6be832106/local_build.py}} by our fellow \href{https://gist.github.com/eteq}{Erik Tollerud}. 